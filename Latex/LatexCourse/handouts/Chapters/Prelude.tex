\chapter{Prelude}
\label{chap:prelude}
\section{What is \LaTeX\ and how does it work?}
\LaTeX\ is a typesetting program. It generates a professionally looking and easily printable document (e.g.~a {\tt pdf} file) from user provided input files. The main input is a {\tt tex} file, which contains the source code --- raw text interspersed with commands. The commands basically tell \LaTeX\ what to do with the text. For example, you can make text \textbf{bold}, \emph{italic}, or \underline{underlined} by first typing {\textbackslash}textbf\{\}, {\textbackslash}emph\{\} or {\textbackslash}underlined\{\}, respectively, and then placing the text inside the curly brackets.

The typesetting rules applied by \LaTeX\ are targeted mainly at aesthetics, and these rules have been honed by professional typesetters since the time of Caxton!

\section{Why bother with it?}
\LaTeX\ is not the ``magic bullet'' of typesetting --- it has pros and cons --- but there are good reasons why \LaTeX\ is widely used in scientific publishing and academia.  
\subsection{Advantages}
\begin{itemize}
\item It's free of charge.
\item We can use any text editor to view and modify the source code.
\item More time is spent on creating content and less time on cosmetic tweaking.
\item Basic typesetting is done automatically. \LaTeX\ uses consistent rules throughout a document, making it look professional.
\item Precise changes can be introduced globally with very little effort.
\item Mathematical equations, like $E=mc^2$ or $\imath\hbar\frac{\partial}{\partial t}\Phi (x, t) = \hat{H}\Phi (x, t)$ can be produced almost as fast as typing (if you know the commands!).
\end{itemize}

\subsection{Disadvantages}
\begin{itemize}
\item Steep learning curve.
\item You don't see the output as you go.
\item Sometimes \LaTeX\ does not work quite the way you want it to, and learning how to influence it can be a challenge.
\item Errors in the source code can be surprisingly difficult to locate.
\end{itemize}

\subsection{\LaTeX\ versus MS Word/Office}
\begin{itemize}
\item Large technical documents are often much easier to handle in \LaTeX. 
\item In \LaTeX, technical features such as tables, equations and figures are integrated much more smoothly.
\item It is generally quicker to write and debug the \LaTeX\ source than it is to typeset an entire thesis manually and consistently in Word/Office.
\end{itemize}


%Although intended to save work the principle of `conservation of work' means that you simply transform problems associated with WYSIWYG approaches to problems associated with WYSIWYM approaches!

\section{\LaTeX\ distributions and editors}
\LaTeX\ comes in multiple flavours/distributions, with the most common ones being:
\begin{itemize}
\item TeXLive for Windows and Unix/Linux.
\item MacTeX for Mac OS.
\end{itemize}
You also need to choose an editor (sometimes referred to as ``the front end'') for writing the source code.
I will be using gedit, a basic text editor, during this workshop. Several more advanced front ends are
available on MCS Linux. All of them provide an editing environment, but use calls to the system's
TeXLive distribution.


