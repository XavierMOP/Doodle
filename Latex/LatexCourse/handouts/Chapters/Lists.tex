\chapter{Lists}
Lists are great.  The command {\textbackslash}begin\{\} can be used to enter a list environment.  For example:

\begin{verbatim}
\begin{itemize}
\item cat
\item dog
\item horse
\end{itemize}
\end{verbatim}

Produces:

\begin{itemize}
\item cat
\item dog
\item horse
\end{itemize}

\pagebreak
We can also replace the bullet points with numbers using the enumerate keyword.

\begin{verbatim}
\begin{enumerate}
\item cat
\item dog
\item horse
\end{enumerate}
\end{verbatim}

\begin{enumerate}
\item cat
\item dog
\item horse
\end{enumerate}

\pagebreak
and we can use the description keyword which does this:
\begin{verbatim}
\begin{description}
\item[Cat] a lovely furry creature with a cute nose and whiskers.
\item[Dog] Another furry creature that smells rather well; 
           its olfactory power stems from its nasal dampness.
\item [Horse] A large stinky creature with sideways facing eyes.
\end{description}
\end{verbatim}

\begin{description}
\item[Cat] a lovely furry creature with a cute nose and whiskers.
\item[Dog] Another furry creature that smells rather well; its olfactory power stems from its nasal dampness.
\item [Horse] A large stinky creature with sideways facing eyes.
\end{description}